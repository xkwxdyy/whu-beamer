\documentclass{whu-beamer}

% 图片所在路径
\graphicspath{{figures/}{logo/}}
% 信息设置
\whubeamersetup{
  info = {
    author            = {郭鑫},
    advisor           = {王茂发 \, 教授},
    title             = {全纯函数空间上复合算子的差分},
    date              = {2019.5.26},
    research-fields   = {函数空间上的算子理论},
    type              = {博士论文答辩}
  },
  bib = {
    % 参考文献数据库
    resource = {demo.bib}
  }
}

% 加载自己所需宏包
\usepackage{amsmath}



\begin{document}

% 标题页
\maketitle


% 目录
\section{内容提要}

\begin{frame}
  \frametitle{内容提要}
  \tableofcontents[hideallsubsections]
\end{frame}



% 正文
\section{研究背景}
\begin{frame}
  \begin{enumerate}[i]
    \item 1
    \item 2
  \end{enumerate}
  \begin{enumerate}[a]
    \item 1
    \item 2
  \end{enumerate}
\end{frame}

\begin{frame}
  \begin{definition}
    这是一段文字 $E = m c^2$
  \end{definition}
  
  \begin{theorem}
    这是一段文字 $E = m c^2$
  \end{theorem}
  
  
  \begin{proof}
    这是一段文字 $E = m c^2$
  \end{proof}
  
  \begin{proof}[定理xx的证明]
    这是一段文字 $E = m c^2$
  \end{proof}
  
\end{frame}

\begin{frame}
  \begin{example}
    这是一段文字 $E = m c^2$
  \end{example}
  
  \begin{property}
    这是一段文字 $E = m c^2$
  \end{property}
  
  \begin{proposition}
    这是一段文字 $E = m c^2$
  \end{proposition}
  
  \begin{corollary}
    这是一段文字 $E = m c^2$
  \end{corollary}
\end{frame}

  
\begin{frame}
  \begin{lemma}
    这是一段文字 $E = m c^2$
  \end{lemma}
  
  \begin{axiom}
    这是一段文字 $E = m c^2$
  \end{axiom}
  
  \begin{antiexample}
    这是一段文字 $E = m c^2$
  \end{antiexample}
  
  \begin{conjecture}
    这是一段文字 $E = m c^2$
  \end{conjecture}
\end{frame}


\begin{frame}
  \begin{question}
    这是一段文字 $E = m c^2$
  \end{question}
  
  \begin{claim}
    这是一段文字 $E = m c^2$
  \end{claim}
\end{frame}


\begin{frame}
  \begin{definition}[序有界算子]
    设 $\mu$ 为区域 $G$ 上的 Borel 测度, $X$ 为区域 $G$ 上全纯函数组成的 Banach 空间. 称线性算子 $T:X\rightarrow L^{p}(d\mu)$ 为序有界算子,若存在 $g\in L^{p}(d\mu), g\geq0,$ 对于任意的 $f\in X$ 且 $\|f\|_{X}\leq1$ 使得 $|Tf|\leq g$.
  \end{definition}

  \begin{theorem}[MacCluer-Shapiro, 1986]
    复合算子 $C_{\varphi}$ 在加权 Bergman 空间 $A^{p}_{\alpha}(\mathbf{D})$ 为紧算子当且仅当
    \[ 
      \lim_{|z|\rightarrow 1}\frac{1-|z|^{2}}{1-|\varphi(z)|^{2}}= 0.
    \]
  \end{theorem}
\end{frame}



%% 如果有需要显示参考文献的
    % 1. 正文中用 \cite 引用 .bib 数据库文件中的参考文献
    % 2. 取消下面 `\section{参考文献}... \printbibliography \end{frame}` 代码的注释
    % 3. 使用 `xelatex -> biber -> xelatex*2` 的方式进行编译

% \section{参考文献}
% \begin{frame}
%   \frametitle{参考文献}
%   \printbibliography
% \end{frame}



% 致谢
\begin{acknowledgements}[color = blue!80]
  Thanks for your attention!

  感谢各位老师的聆听
\end{acknowledgements}

\end{document}